
\chapter*{ABSTRAK} \addcontentsline{toc}{chapter}{ABSTRAK}
\begin{tabular}{l l p{10cm}}
Nama &:& \Peneliti\\
Program Studi &:& \jur \\
Judul &:& \judul
\end{tabular} 
\vspace{0.3cm}\\
\vspace{0.1cm}\\

Telah dilakukan penelitian tentang gerak berkelompok yang terdapat pada hewan seperti burung, ikan bahkan manusia. Pergerakan dapat diimplementasikan pada tawaf yang merupakan pola yang yang dipengaruhi oleh keinginan untuk berkelompok agar effisien dan mendekat pada jenis yang sama. keiginan tersebut bergantung pada parameter-parameter tertentu sebagai sistem. Untuk memahami sistem ini dapat dilakukan pendekatan partikel. Interaksi antar partikel dapat digolongkan  sebagai keinginan partikel untuk berkelompok sejenis. dan interaksi kedua dipengaruhi oleh gaya dominan yang memmbuat keinginan tawaf tersebut terpola. Pendekatan Newtonian digunakan untuk
menangani parameter fisika seperti posisi dan kecepatan partikel yang Kemudian dihitung menggunakan integrase runge-kutta. Pola yang muncul diantaranya adalah pola gerak menuju pusat gaya(kaaba). \\

\textbf{\textit{Kata Kunci: gerak berkelompok, runge-kutta, gaya dominan }}
