%\input{format}
%\begin{document}
\chapter{PENUTUP}\label{cha:penutup}
\section{Kesimpulan}

\hspace{0.5cm} Berdasarkan hasil penelitian dan study literatur dapat disimpulkan sementara bahwa:

\begin{enumerate} 
\item  Pergerakan tawaf seperti nasir dengan partikel dapat flocking(mencapai kecepatan yang tetap), dapat membentuk group dan mengelilingi Kaaba, sedangkan skema dalam bergerak mendekati hasil dari kim
\item  Pergerakan partikel berkelompok yang dapat mendekati Kaaba(terutama pada partikel hitam dalam optimasi). 
\item  Penggunakan runge-kutta,dikunakan untuk pemedatan data dalam bentuk grafik. Berdasarkan simulasi yang telah dilakukan.
\end{enumerate} 


\newpage
\section{Saran}\label{cha:saran}
% \begin{enumerate}
% \item partikel lain juga mengikuti Gerakan partikel grouping mendekati pusat Kaaba.
% \item Dalam beberapa case partikel sering melakukan resultan gerak berlawanan arah(mungkin perlu gaya tambahan untuk mendukung arah partikel agar mengelilingi).
% \item Skema ruang partikel mungkin harus diperbesar Bersama dengan Kaaba pusatnya hingga mirip dengan paper nasir

% \end{enumerate}
\hspace{0.5cm} Penelitian ini hanya berfokus pada pembuatan mekanisme flocking pada gerak tawaf. Faktor-faktor fisis seperti gesekan partikel dengan medium yang dilalui dan bentuk skema sebelum tawaf terjadi tidak diperhitungkan. Begitu juga dengan object yang ada dalam garis tawaf sebenarnya kecuali kaaba sebagai pusat gaya. 

%\end{document}