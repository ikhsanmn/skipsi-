\chapter*{ABSTRACT} \addcontentsline{toc}{chapter}{ABSTRACT}
\begin{tabular}{l l p{10cm}}
\textit{Name} &:& \Peneliti\\
\textit{Studies Program} &:& \prog\\
\textit{Title} &:& \named
\end{tabular} 
\vspace{0.3cm}\\
\vspace{0.1cm}\\

%\textit{Research has been conducted .}\\
Research has been done on group movement found in animals such as birds, fish and even humans. Movement can be applied to tawaf which is a pattern that is influenced by the desire to group together to be efficient and closer to the same type. the desire depends on certain parameters as a system. To understand this system, a particle approach can be used. The interaction between particles can be classified as the desire of particles to group together. and the second interaction is influenced by the dominant force that makes the tawaf desire patterned. The Newtonian approach is used to
handles physical parameters such as position and velocity of the particles which are then calculated using the runge-kutta integration. The patterns that emerge include the pattern of motion towards the center of force (kaaba). \\

\textbf{\textit{Keywords: group movement, runge-kutta, dominant force }}