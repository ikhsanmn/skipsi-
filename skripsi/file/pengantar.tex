%%\begin{document}
%\setverb\setnash
%\hyphenpenalty=10000
%\sloppypar
\chapter*{KATA PENGANTAR} \addcontentsline{toc}{chapter}{KATA PENGANTAR}

Segala puji bagi Allah SWT, \model \ yang berjudul \judul\ dapat diselesaikan. Ini diajukan untuk memenuhi syarat kelulusan program Strata-1 (S1) jurusan \jur, \fak, \kampus. 
\paragraph{}
\model \ ini selesai dengan adanya bantuan dari berbagai pihak. Oleh karena itu penulis ucapkan terima kasih yang kepada:
\begin{enumerate}
\item Allah SWT yang selalu berada di dekat penulis memberikan rahmat dan hidayah untuk menjadi manusia yang berguna.
\item Nabi Muhammad SWA yang mana beliu adalah panutan bagi penulis.
% \item Bapak DR.Eng.Bagus Endar Nurhandoko, Selaku Direktur Utama PT Rock Fluid Imaging Lab yang telah memberikan kesempatan kepada penulis untuk melaksanakan penelitian Tugas Akhir di Perusahaan PT Rock Fluid Imaging Lab.
% \item Bapak kaswandhi Triyoso.,M.Si dan Bapak Muhammad Rizka Asmara Hadi.,S.T Selaku Dosen Pembimbing Tugas Akhir PT Rock Fluid Imaging Lab, karena atas Bimbingan dan kepercayaan yang beliau berikan kepada penulis untuk menyelesaikan skripsi ini.
% \item Bapak Mada Senjaya W.S., Ph.D Selaku Dosen Pembimbing Jurusan Fisika, karena atas Bimbingan dan kepercayaan yang beliau berikan kepada penulis untuk menyelesaikan skripsi ini.
% \item Bapak Dr.rar.nat. Imamal Mutaqqien.M.Si Selaku Dosen Penguji 1 dan Pembimbing keahlian Fisika Bumi
% \item Bapak Dr.Yudha Setya Perkasa.M.Si, selaku Dosen Penguji 2 dan selaku ketua Jurusan Fisika.    
\item Bapak Dr.Yudha Setya Perkasa.M.Si Sebagai Dosen Pembimbing Akademik Jurusan Fisika, karena atas Bimbingan dan kepercayaan yang beliau berikan kepada penulis untuk menyelesaikan Akademik di Jurusan Fisika  ini.
\item Bapak Ridwan Ramdani, S.Si., M.Si Selaku Dosen Pembimbing Jurusan Fisika, karena atas Bimbingan dan kepercayaan yang beliau berikan kepada penulis untuk menyelesaikan skripsi ini.
\item Seluruh dosen fisika UIN Bandung yang telah banyak meluangkan waktunya untuk memberikan pengetahuan, arahan, masukan, dan dukungan yang berarti bagi penulis. Dan juga senantiasa membimbing penulis mempelajari keilmuan dalam menyelesaikan skripsi ini.
\item Rekan-rekan Fisika angkatan 2015 yang memotivasi  agar penulis lulus sebagai sarjana. 
\item Mahasiswa fisika angkatan 2008, 2009, 2010, 2011, 2012, 2013, 2014 ,2015 ,2016 ,2017 yang selalu memberikan dukungan dan motivasi agar penulis lulus sebagai sarjana.
\item Rekan seperjuangan laboratorium sistem modeling Dinda, Ariq, Arum dan Indri.
% \item Pimpinan dan Keluarga Besar Mahad Tahfidz Al-Quran Al-Kausar 561 Kampung Jagabaya Desa Kecamatan Cineam Kabupaten Tasikmalaya yang telah memberikan tempat penelitian buat penulis untuk melaksanakan penelitian tugas akhir.
% \item Orang tua Bapak H.Yayan Hadian.S.Pd,M.Si dan Ibu Euis Susi Nurhayati.S.Pd.I, adik-adikku Nasiha Al-Sakinah dan Amada Laila Fitria, serta keluarga H.Endang dan H.A Bahrum Sujai yang telah memberikan perhatian, bantuan, motivasi dan pengertian serta doa restu yang penulis sadari mempunyai peran penting dalam menyelesaikan Akademik sehingga penulis lulus sebagai sarjana.
% \item Keluarga kelompok  Desa Sukamatri Kecamatan Tarogong Kaler Kabupaten Garut (KKN) yang pernah bersama-sama hidup selama satu bulan dan memberikan kecerian yang luar biasa. 
\end{enumerate}
\newpage
Akhir kata, penulis menyadari karya tulis ini masih banyak kekurangan oleh karena itu kritik dan saran yang membangun sangat diharapkan demi kesempurnaan skripsi ini. dengan harapan semoga karya tulis ini bermanfaat bagi pembaca khususnya yang mendalami model komputasi.
 
\vspace{0.5cm}
\begin{flushright}
\begin{tabular}{c}
Bandung, \waktu
\\\\\\
Penulis
\end{tabular}
\end{flushright}
%\end{document}