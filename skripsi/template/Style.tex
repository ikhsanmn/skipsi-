\documentclass[12pt,a4paper]{report}
\usepackage[left=4.00 cm,top=4 cm,right=3.00 cm,bottom=3.00cm]{geometry}
\usepackage[bahasa]{babel}
\usepackage[hang]{caption2}  % bisa, tapi tidak ada untuk bahasa arab
%\usepackage{arabtex}
\usepackage{apacite, natbib}
%\usepackage{relsize}
\usepackage{fancyhdr} 
\pagestyle{fancy}
\usepackage{setspace} %\doublespacing 
\usepackage{verbatim}
\usepackage{graphicx}
\usepackage{subfigure}
\usepackage{rotating} 
\usepackage{hyperref} 
\usepackage{titlesec,wrapfig,float,eso-pic,makeidx,pst-node,float,hyperref,eso-pic,titlesec,pst-node,pstricks}
\usepackage{amsmath}
\usepackage{multirow}
\bibliographystyle{seg}
%\usepackage{arabtex}

%%%%%%%%%%%%%%%%%%%%%%%%%%%%%%%%%%%%%%%%%%%%%%%%%%%%%%%%%%%%%%%%%%%%%%%%%%%%%%%%%%%%%%%%%%%%%%%%%%%%%%%%%%%%%%%
\newcommand{\var}[2]{\newcommand{#1}{#2}}
\newcommand{\Var}[2]{\newcommand{#1}{\uppercase{#2}}}

\var{\peneliti}{ahmad rasyid hilmansyah}
\var{\Peneliti}{AHMAD RASYID HILMANSYAH}
\var{\nim}{1137030003}
\var{\judul}{Perbandingan Metode Geolistrik Konfigurasi Wenner-Schlumberger Dengan Wenner Alpha Dalam Pendugaan Potensi Air Tanah  Di Mahad Tahfidz Al-Quran Al-kausar 561 Kecamatan Cineam Kabupaten Tasikmalaya}
\Var{\Judul}{PERBANDINGAN METODE GEOLISTRIK KONFIGURASI WENNER-SCHLUMBERGER DENGAN WENNER ALPHA DALAM PENDUGAAN POTENSI AIR TANAH  DI MAHAD TAHFIDZ AL-QURAN AL-KAUSAR 561 KECAMATAN CINEAM KABUPATEN TASIKMALAYA}
\var{\named}{\textit{Comparison of geoelectric methods wenner-schlumberger configuration with wenner alpha in estimating groundwater potential in mahad tahfidz al-quran al-Kausar 561 cineam district Tasikmalaya district}}
\var{\prog}{Physics}
\var{\jur}{Fisika}
\Var{\Jur}{FISIKA}
\var{\fak}{Fakultas Sains dan Teknologi}
\Var{\Fak}{FAKULTAS SAINS DAN TEKNOLOGI}
\var{\kampus}{UIN Sunan Gunung Djati Bandung}
\Var{\Kampus}{UNIVERSITAS ISLAM NEGERI SUNAN GUNUNG DJATI BANDUNG}
\var{\tahun}{2018}
%\var{\waktu}{\today}
\var{\waktu}{27 Agustus 2018}

\var{\model}{Skripsi}
\Var{\Model}{SKRIPSI}

\var{\alat}{}
\var{\Alat}{}
\author{\peneliti}
\title{Comparison of geoelectric method of the configuration of wenner-schlumberger with wenner alpha in the estimation of groundwater potential in mahad tahfidz al-quran al-kausar 561 kecamatan cineam tasikmalaya district}
%\title{\centering}
%%%%%%%%%%%%%%%%%%%%%%%%%%%%%%%%%%%%%%%%%%%%%%%%%%%%%%%%%%%%%%%%%%%%%%%%%%%%%%%%%%%%%%%%%%%%%%%%%%%%%%%%%%%%%%%
% untuk rumus
\usepackage[T1]{fontenc}
\usepackage{color}
\usepackage{alltt}
\usepackage{times}
\usepackage{ulem}
\usepackage[ansinew]{inputenc}
\usepackage{url}
\usepackage{lscape}



% Untuk diagram alir
\usepackage{tikz}
\usetikzlibrary{calc,trees,positioning,arrows,chains,shapes.geometric,decorations.pathreplacing,decorations.pathmorphing,shapes,matrix,shapes.symbols,arrows}

\tikzstyle{startstop} = [rectangle, rounded corners, minimum width=3cm, minimum height=1cm,text centered, draw=black] %fill=red!30]

\tikzstyle{io} = [trapezium, trapezium left angle=70, trapezium right angle=110, minimum width=3cm, minimum height=1cm, text centered, draw=black,]% fill=blue!30]

\tikzstyle{process} = [rectangle, minimum width=3cm, minimum height=1cm, text centered, draw=black] %fill=orange!30]

\tikzstyle{processs} = [rectangle,  minimum height=1cm,text width=3.4cm, draw=black] %minimum width=4cm,

\tikzstyle{decision} = [diamond, minimum height=1cm, text centered, draw=black] %fill=green!30, minimum width=3cm,]

\tikzstyle{arrow} = [thick,->,>=stealth]
\tikzstyle{win} = [rectangle, rounded corners, minimum width=3cm, minimum height=1cm,text centered, draw=black, fill=red!30]
\tikzstyle{to} = [trapezium, trapezium left angle=70, trapezium right angle=110, minimum width=3cm, minimum height=0.7cm, text centered, draw=black, fill=green!30]
\tikzstyle{proses} = [rectangle, minimum width=3cm, minimum height=0.7cm, text centered, draw=black, fill=blue!20]
\tikzstyle{pilihan} = [diamond, text width=2cm,text centered, draw=black, fill=orange!30]
\tikzstyle{garis} = [thick,->,>=stealth]
\tikzstyle{kres} = [rectangle, rounded corners, minimum width=2cm, minimum height=0.5cm,text centered, draw=black]
\tikzstyle{nad} = [trapezium, trapezium left angle=70, trapezium right angle=110, minimum width=3cm, minimum height=0.5cm, text centered, draw=black]
\tikzstyle{jaja} = [rectangle, minimum width=2cm, minimum height=0.5cm, text centered, draw=black]
\tikzstyle{gan} = [diamond, minimum width=0.5cm, minimum height=0.5cm, text centered, draw=black]

%tulisan berwarna
\newgray{abu}{0.3}
\newcmykcolor{koneng}{0 0.8 0.8 0.1}
\newcmykcolor{hejo}{1 0 1 0.5}
\newrgbcolor{biru}{0 1 1}


\allowdisplaybreaks[1]
\usepackage[ConnyRevised]{fncychap}
\titleformat{\chapter}[display] {\normalfont\huge\bfseries\centering} {BAB  \thechapter}{18pt}{\LARGE}
\onehalfspacing
\floatplacement{figure}{H}  \floatplacement{table}{H}
\hypersetup{pdfborder=0 0 0, colorlinks=black, linkcolor=black, citecolor=black, bookmarksopen=true, bookmarksnumbered=true, pdfstartview=FitH, pdfview=FitH} 


\lhead{}%\nouppercase{\rightmark}
\chead{}
\rhead{}

\cfoot{\thepage}
\rfoot{}
\renewcommand{\headrulewidth}{0.0pt}


\def\captionsbahasa{
%\def\prefacename{KATA PENGANTAR}%
%\def\contentsname{DAFTAR ISI}%
%\def\listfigurename{DAFTAR GAMBAR}%
%\def\listtablename{DAFTAR TABEL}%
%\def\listappendixname{DAFTAR LAMPIRAN}%
%\def\nomenclaturename{DAFTAR SINGKATAN}%
%\def\abstractname{Intisari}%
%\def\acknowledgmentname{HALAMAN PERSEMBAHAN}%
%\def\approvalname{HALAMAN PENGESAHAN}
%\def\partname{BAGIAN}%
%\def\chaptername{BAB}%
\def\appendixname{LAMPIRAN}%
%\def\refname{DAFTAR PUSTAKA}%
\def\bibname{DAFTAR PUSTAKA}%
\def\indexname{INDEX}%
\def\figurename{\textbf{Gambar}}%
\def\tablename{\textbf{Tabel}}%
\def\pagename{Halaman}
%tambahan
\def\rumus{\textbf{Persamaan}}
\def\gambar{\figurename}
\def\tabel{\tablename}
\def\lampiran{\textbf{Lampiran}}


%dosen jurusan fisika
\def\mada{\underline{\textbf{Mada Sanjaya W.S., Ph.D}}}
\def\nipmada{\textbf{198510112009121005}}
\def\yudha{\underline{\textbf{Dr. Yudha Satya Perkasa}}}
\def\nipyudha{\textbf{197911172011011005}}
\def\bebeh{\underline{\textbf{Dr. Bebeh Wahid Nuryadin}}}
\def\nipbebeh{\textbf{19860816201101109}}
\def\kaswandhi{\underline{\textbf{Kaswandhi Triyoso, M.Si}}}
\def\nipkaswandhi{\textbf{198510172007070105}}
\def\imamal{\underline{\textbf{Dr. rer.nat. Imamal Muttaqien}}}
\def\nipimamal{\textbf{198310062009121009}}
%dekanat
\def\dekan{\underline{\textbf{Dr. H. Opik Taupik Kurahman}}}
\def\nipdekan{\textbf{196812141996031001}}
}
\normalem

\makeatletter
\newsavebox\zzz
\def\mystrut{%
\dimen@\wd\zzz
\divide\dimen@\thr@@
\advance\dimen@-\dp\@arstrutbox
\rule\z@\dimen@}

\def\rotatezzz{%
\rotatebox{0}{\rlap{\kern-\dp\@arstrutbox\usebox\zzz}}}

\makeatother