\documentclass[twocolumn,11pt]{article}
\usepackage{times}
%
% DO NOT CHANGE THE FOLLOWING PART
%
\setlength{\textwidth}{6.9in}
\setlength{\textheight}{9.5in}
\setlength{\oddsidemargin}{-0.25in}
\setlength{\evensidemargin}{-0pt}
\setlength{\topmargin}{-0.25in}
\setlength{\columnsep}{0.4in}
\setlength{\parindent}{4ex}
%
\newtheorem{definition}{Definition}
\newtheorem{remark}[definition]{Remark}
\newtheorem{lemma}[definition]{Lemma}
\newtheorem{theorem}[definition]{Theorem}
\newtheorem{proposition}[definition]{Proposition}
\newtheorem{corollary}[definition]{Corollary}
%
%
% THIS IS THE PLACE FOR YOUR OWN DEFINITIONS
%
\newcommand{\R}{{\rm I}\!{\rm R}} % the set of real numbers
% boldface characters in mathematical formulas
\newcommand{\bff}{\mbox{\boldmath $f$}}
\newcommand{\bfl}{\mbox{\boldmath $l$}}
\newcommand{\bfn}{\mbox{\boldmath $n$}}
\newcommand{\bfu}{\mbox{\boldmath $u$}}
\newcommand{\bfx}{\mbox{\boldmath $x$}}
\newcommand{\bfy}{\mbox{\boldmath $y$}}
\newcommand{\bfzero}{\mbox{\boldmath $0$}}
\newcommand{\bfcurl}{{\bf curl}} % curl of a vector field
\newcommand{\rmdiv}{{\rm div}} % divergence of a vector field
\newcommand{\eop}{\hfill $\sqcap\!\!\!\!\sqcup$} % end of proof
%
%
% THE BEGINNING OF THE DOCUMENT
%